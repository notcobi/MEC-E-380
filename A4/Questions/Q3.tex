\section{}
%At a point in a stressed body, the strains, related to the coordinate set xyz, are given by Eqn. (1).

At a point in a stressed body, the strains, related to the coordinate set $xyz$, are given by:
\begin{equation}
    \epsilon = 
    \begin{bmatrix}
        \epsilon_{x} & \frac{1}{2} \gamma_{xy} & \frac{1}{2} \gamma_{xz} \\
        \frac{1}{2} \gamma_{xy} & \epsilon_{y} & \frac{1}{2} \gamma_{yz} \\
        \frac{1}{2} \gamma_{xz} & \frac{1}{2} \gamma_{yz} & \epsilon_{z}
    \end{bmatrix}
    = 
    \begin{bmatrix}
        400 & 100 & 0 \\
        100 & 0 & -200 \\
        0 & -200 & 600  
    \end{bmatrix}
    \times 10^{-6}
\end{equation}

Determine,
\begin{enumerate}[label=(\alph*)]
    \item the strain invariants.
    \item the normal strain in the $x'$ direction, which is directed at an angle $30^\circ$ from the $x$-axis.
    \item the principal strains $\epsilon_1$, $\epsilon_2$, and $\epsilon_3$.
    \item the maximum shear strain.
\end{enumerate}

\subsection{}
Determine $I_1$, $I_2$, and $I_3$.
\begin{align*}
    I_1 &= \epsilon_{x} + \epsilon_{y} + \epsilon_{z} \\
    &= 400 + 0 + 600 \\
    &= \boxed{1.00 \times 10^{-3}} \\
    I_2 &= \epsilon_{x}\epsilon_{y} + \epsilon_{y}\epsilon_{z} + \epsilon_{z}\epsilon_{x} - \gamma_{xy}^2 - \gamma_{yz}^2 - \gamma_{xz}^2 \\
    &= 400(0) + 0(-200) + 600(400) - 100^2 - (-200)^2 - 0^2 \\
    &= \boxed{1.90 \times 10^{-7}} \\
    I_3 &= det(\epsilon) \\
    & = \boxed{-2.2 \times 10^{-11}}
\end{align*}

\subsection{}
The normal strain $\epsilon_{x'}$ in the direction of $\theta = 30^\circ$ is given by:
\begin{align*}
    \epsilon_{x'} &= \epsilon_x\cos^2\theta + \epsilon_y\sin^2\theta + \gamma_{xy}\sin\theta\cos\theta \\
    &= 400\cos^2(30^\circ) + 0\sin^2(30^\circ) + 2(100)\sin(30^\circ)\cos(30^\circ) \\
    &= \boxed{\qty{386e-6}{}}\\
\end{align*}

\subsection{}
The characteristic equation for the principal strains is given by:
\begin{align*}
    \lambda^3 - I_1\lambda^2 + I_2\lambda - I_3 &= 0 \\
\end{align*}

Plugging these into polynomial solver yields:

\begin{equation*}
    \boxed{\epsilon = 6.64 \times 10^{-4}, \; 4.16 \times 10^{-4}, \; -7.97 \times 10^{-5}}
\end{equation*}

\subsection{}
The maximum shear strain is given by the difference between the maximum and minimum principal strains:
\begin{align*}
    \gamma_{max} &= \epsilon_1 - \epsilon_3 \\
    &= \boxed{\qty{7.44e-4}{}}
\end{align*}