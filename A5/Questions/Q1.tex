\section{}The distribution of stress in an aluminum machine component is given by:
\[
\sigma = 
\begin{bmatrix}
    \sigma_x & \tau_{xy} & \tau_{xz} \\
    \tau_{yx} & \sigma_y & \tau_{yz} \\
    \tau_{zx} & \tau_{zy} & \sigma_z
\end{bmatrix}
=
\begin{bmatrix}
    y + 2z^2 & 3z^2 & 2y^2 \\
    3z^2 & x + z & x^2 \\
    2y^2 & x^2 & 3x + y
\end{bmatrix}
\si{\mega\pascal}
\]

Calculate the state of strain of a point positioned at $(1, 2, 4)$. Use $E = \qty{70}{GPa}$ and $\nu = 0.3$.

\textbf{Solution}

From the generalized Hooke's law, the stress-strain relation is given by:
\[
\begin{aligned}
    \epsilon_x &= \frac{1}{E} (\sigma_x - \nu(\sigma_y + \sigma_z)) \\
    \epsilon_y &= \frac{1}{E} (\sigma_y - \nu(\sigma_x + \sigma_z)) \\
    \epsilon_z &= \frac{1}{E} (\sigma_z - \nu(\sigma_x + \sigma_y)) \\
    \gamma_{xy} &= \frac{1}{2G} \tau_{xy} \\
    \gamma_{yz} &= \frac{1}{2G} \tau_{yz} \\
    \gamma_{xz} &= \frac{1}{2G} \tau_{xz}
\end{aligned}
\]

Where 
\[
\begin{aligned}
    G &= \frac{E}{2(1 + \nu)}
    &= \frac{70}{2(1 + 0.3)}
    &= \qty{26.923}{GPa}
\end{aligned}
\]

Evaluate the stress state at $(1, 2, 4)$:
\[
\begin{aligned}
    \sigma &= 
    \begin{bmatrix}
        2 + 2(4)^2 & 3(4)^2 & 2(2)^2 \\
        3(4)^2 & 1 + 4 & 1^2 \\
        2(2)^2 & 1^2 & 3(1) + 2
    \end{bmatrix}
    &=
    \begin{bmatrix}
        34 & 48 & 8 \\
        48 & 5 & 1 \\
        8 & 1 & 5
    \end{bmatrix}
    \si{\mega\pascal}
\end{aligned}
\]

Using the stress-strain relations, the strain state is given by:
\[
\begin{aligned}
    \epsilon &=
    \begin{bmatrix}
        \frac{1}{E} (\sigma_x - \nu(\sigma_y + \sigma_z)) & \frac{1}{2G} \tau_{xy} & \frac{1}{2G} \tau_{xz} \\
        \frac{1}{2G} \tau_{yx} & \frac{1}{E} (\sigma_y - \nu(\sigma_x + \sigma_z)) & \frac{1}{2G} \tau_{yz} \\
        \frac{1}{2G} \tau_{zx} & \frac{1}{2G} \tau_{zy} & \frac{1}{E} (\sigma_z - \nu(\sigma_x + \sigma_y))
    \end{bmatrix} \\
    &=
    \begin{bmatrix}
        \frac{1}{70\times 10^3} (34 - 0.3(5 + 5)) & \frac{1}{2(26.923)\times 10^3} 48 & \frac{1}{2(26.923)\times 10^3} 8 \\
        \frac{1}{2(26.923)\times 10^3} 48 & \frac{1}{70\times 10^3} (5 - 0.3(34 + 5)) & \frac{1}{2(26.923)\times 10^3} 1 \\
        \frac{1}{2(26.923)\times 10^3} 8 & \frac{1}{2(26.923)\times 10^3} 1 & \frac{1}{70\times 10^3} (5 - 0.3(34 + 5))
    \end{bmatrix} \\
    &=
    \begin{bmatrix}
        4.43 \times 10^{-4} & 8.91 \times 10^{-4} & 1.49 \times 10^{-4} \\
        8.91 \times 10^{-4} & -9.57 \times 10^{-5} & 1.86 \times 10^{-5} \\
        1.49 \times 10^{-4} & 1.86 \times 10^{-5} & -9.57 \times 10^{-5}
    \end{bmatrix}
\end{aligned}
\]