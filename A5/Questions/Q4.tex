\section{}
% The stress field in an elastic body is given by
% σ =
% [
% σx τxy
% τyx σy
% ]
% =
% [
% cy2 0
% 0 −cx2
% ]
% where c is a constant. Derive expressions for the displacement components u (x, y) and v (x, y) in
% the body.
% 4

The stress field in an elastic body is given by
\[
\sigma =
\begin{bmatrix}
    \sigma_x & \tau_{xy} \\
    \tau_{yx} & \sigma_y
\end{bmatrix}
=
\begin{bmatrix}
    cy^2 & 0 \\
    0 & -cx^2
\end{bmatrix}
\]

Where $c$ is a constant. Derive expressions for the displacement components $u(x, y)$ and $v(x, y)$ in the body.

\textbf{Solution}
\subsection{}
First convert the stresses into strains using the generalized Hooke's law:
\[
\begin{aligned}
    \epsilon_x &= \frac{1}{E} (\sigma_x - \nu(\sigma_y)) \\
    &= \frac{1}{E} (cy^2 - \nu(-cx^2)) \\
    &= \frac{cy^2 + \nu cx^2}{E} \\
    \epsilon_y &= \frac{1}{E} (\sigma_y - \nu(\sigma_x)) \\
    &= \frac{1}{E} (-cx^2 - \nu(cy^2)) \\
    &= \frac{-cx^2 - \nu cy^2}{E} \\
\end{aligned}
\]

Where $E$ is the Young's modulus and $\nu$ is the Poisson's ratio. The strains are related to the displacements by:

\[
\begin{aligned}
    \epsilon_x &= \frac{\partial u}{\partial x}
    \implies u &= \int \epsilon_x dx \\
    &= \int \frac{cy^2 + \nu cx^2}{E} dx \\ 
    &= \boxed{\frac{1}{E} \left(cxy^2 + \frac{\nu}{3} cx^3\right) + g(y)} \\
    \epsilon_y &= \frac{\partial v}{\partial y}
    \implies v &= \int \epsilon_y dy \\
    &= \int \frac{-cx^2 - \nu cy^2}{E} dy \\
    &= \boxed{\frac{1}{E} \left(-cxy^2 - \frac{\nu}{3} cy^3\right) + h(x)} \\
\end{aligned}
\]