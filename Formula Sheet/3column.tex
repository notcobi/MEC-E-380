%%%%%%%%%%%%%%%%%%%%%%%%%%%%%%%%%%%%%%%%%%%%%%%%%%%%%%%%%%%%%%%%%%%%%%
% writeLaTeX Example: A quick guide to LaTeX
%
% Source: Dave Richeson (divisbyzero.com), Dickinson College
% 
% A one-size-fits-all LaTeX cheat sheet. Kept to two pages, so it 
% can be printed (double-sided) on one piece of paper
% 
% Feel free to distribute this example, but please keep the referral
% to divisbyzero.com
% 
%%%%%%%%%%%%%%%%%%%%%%%%%%%%%%%%%%%%%%%%%%%%%%%%%%%%%%%%%%%%%%%%%%%%%%
% How to use writeLaTeX: 
%
% You edit the source code here on the left, and the preview on the
% right shows you the result within a few seconds.
%
% Bookmark this page and share the URL with your co-authors. They can
% edit at the same time!
%
% You can upload figures, bibliographies, custom classes and
% styles using the files menu.
%
% If you're new to LaTeX, the wikibook is a great place to start:
% http://en.wikibooks.org/wiki/LaTeX
%
%%%%%%%%%%%%%%%%%%%%%%%%%%%%%%%%%%%%%%%%%%%%%%%%%%%%%%%%%%%%%%%%%%%%%%

\documentclass[letter]{article}
\usepackage{amssymb,amsmath,amsthm,amsfonts}
\usepackage{multicol,multirow}
\usepackage{calc}
\usepackage{ifthen}
\usepackage[landscape]{geometry}
\usepackage[colorlinks=true,citecolor=blue,linkcolor=blue]{hyperref}
\usepackage{graphicx}
\usepackage{float}
\usepackage{enumitem}
\usepackage{nccmath}
\usepackage[font=footnotesize]{caption}
\usepackage{pdfpages}

\ifthenelse{\lengthtest { \paperwidth = 11in}}
    { \geometry{top=.5in,left=.5in,right=.5in,bottom=.5in} }
	{\ifthenelse{ \lengthtest{ \paperwidth = 297mm}}
		{\geometry{top=1cm,left=1cm,right=1cm,bottom=1cm} }
		{\geometry{top=1cm,left=1cm,right=1cm,bottom=1cm} }
	}

\pagestyle{empty}
\makeatletter
\renewcommand{\section}{\@startsection{section}{1}{0mm}%
                                {-1ex plus -.5ex minus -.2ex}%
                                {0.5ex plus .2ex}%x
                                {\normalfont\large\bfseries}}
\renewcommand{\subsection}{\@startsection{subsection}{2}{0mm}%
                                {-1explus -.5ex minus -.2ex}%
                                {0.5ex plus .2ex}%
                                {\normalfont\normalsize\bfseries}}
\renewcommand{\subsubsection}{\@startsection{subsubsection}{3}{0mm}%
                                {-1ex plus -.5ex minus -.2ex}%
                                {1ex plus .2ex}%
                                {\normalfont\small\bfseries}}

\makeatother
\setcounter{secnumdepth}{0}
\setlength{\parindent}{0pt}
\setlength{\parskip}{0pt plus 0.5ex}
% -----------------------------------------------------------------------

\title{MEC E 380 Final Formula Sheet}

\begin{document}
% \includepdf[pages=-]{properties_of_some_plane_areas.pdf}
% \includepdf[pages=-]{elementary_formulas.pdf}
\raggedright
\footnotesize

\begin{center}
     \Large{\textbf{MEC E 380 Quiz 4 Formula Sheet}} \\
\end{center}
\begin{multicols}{3}
\setlength{\premulticols}{1pt}
\setlength{\postmulticols}{1pt}
\setlength{\multicolsep}{1pt}
\setlength{\columnsep}{2pt}

\section*{10. Energy Methods}
Castigliano's Theorem:
Displacement
\begin{equation*}
    \delta_i = \frac{1}{EI} \int M_i \frac{\partial M_i}{\partial P_i}dx
\end{equation*}
where $P_i$ is a (dummy) concentrated load.

Angle 
\begin{equation*}
    \delta_i = \frac{1}{EI} \int M_i \frac{\partial V_i}{\partial C_i}dx
\end{equation*}
where $C_i$ is a (dummy) concentrated moment.

For polar coordinates, recall
\begin{equation*}
    \delta_i = \frac{1}{EI} \int M_i \frac{\partial M_i}{\partial P_i}rdrd\theta
\end{equation*}

\input{quiz4.tex}
\section*{5. Bending of Beams}
\subsection*{5.1. General Procedure}
General procedure of asymmetric bending problems
\begin{enumerate}
    \item Identify the location of the centroid of the cross-section, and define it as the origin of the $(y, z)$ coordinate system.
    If the centroid is unknown, set an arbitrary origin and use parallel axis theorem to find the centroid.
    \item Define the orientation of $(y, z)$ axes of the cross-section wisely so that all required moments of 
    inertia $I_y$, $I_z$, and $I_{yz}$ can be obtained (from Table) or calculated easily.
    \item Determine bending moments $M_z$ and $M_y$ at your cross-section. Use elementary beam theory to find the bending moments 
    if given a load.
    \item Use the relations to find the stress $\sigma_x$ and the neutral axis.
\end{enumerate}
\subsection*{5.2. Formulas}
Centroid equations:
\begin{align*}
    \bar{x} &= \frac{\sum \bar{x}_i A_i}{\sum A_i} 
\end{align*}
where $\bar{x}_i$ is the $x$-coordinate of the centroid of the $i$-th area, and $A_i$ is the area of the $i$-th area.

Moment equations:
\begin{gather*}
    M_y = P_z L \\
    M_z = P_y L 
\end{gather*}
where $P_z$ and $P_y$ are positive in the positive $z$ and $y$ directions, respectively.
Parallel axis theorem:
\begin{gather*}
    \bar{z} = \frac{\sum \bar{z}_i A_i}{\sum A_i} \\
    \bar{y} = \frac{\sum \bar{y}_i A_i}{\sum A_i} \\
    I_z = \sum(I_{\bar{z}, i} + A_i d_{y, i}^2) \\
    I_y = \sum(I_{\bar{y}, i} + A_i d_{z, i}^2) \\
    I_{yz} = \sum(I_{\bar{yz}, i} + A_i d_{y, i} d_{z, i}) 
\end{gather*}
where $I_{\bar{z}, i}$, $I_{\bar{y}, i}$, and $I_{\bar{yz}, i}$ are the moments of inertia about the centroidal axes, 
and $d_{y, i}$ and $d_{z, i}$ are the distances from the centroidal axes to the parallel axes. Note:
$I_{yz} = 0$ if there is symmetry about \textbf{either} the $y$ or $z$ direction.

Moment to stress:
\begin{gather*}
    \tau = \frac{VQ}{Ib} \overset{\text{rect}}{=} \frac{3V}{2A_c} \\
    \sigma_{x} = \frac{(M_y I_z + M_z I_{yz})d_{z} - (M_y I_{yz} + M_z I_y)d_{y}}{I_y I_z - I_{yz}^2} \\
    \tan{\phi} = \frac{M_y I_z + M_z I_{yz}}{M_z I_y + M_y I_{yz}} 
\end{gather*}
stress is maximum at the furthest point from the neutral axis on the cross-section. For $\sigma_x$, 
$d_y$ and $d_z$ are the signed displacements ($\pm$) from the centroid to the point of interest in the $y$ and $z$ directions.

Cheesy 


\end{multicols}

\end{document}
